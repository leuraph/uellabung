In this exercise, you will prove the Riemann hypothesis.

\begin{parts}
    \part
    In this section, we'll take the first steps towards 
    unravelling the mystery of the Riemann Hypothesis. 
    Remember, Rome wasn't built in a day, 
    and neither was the proof of the Riemann Hypothesis!
    \begin{subparts}
        \subpart
        Prove that all non-trivial zeros of the Riemann zeta function 
        lie on the critical line. Don't worry, 
        these zeros won't vanish like your hopes of a stress-free proof!
        \subpart
        Now, let's delve into the fascinating world of prime numbers. 
        Show that the distribution of primes is like a puzzle waiting 
        to be solved, with each piece intricately connected to 
        the behaviour of the Riemann zeta function.
    \end{subparts}
    \part
    Welcome to the next chapter of our Riemannian adventure! 
    In this section, we'll continue our quest for the elusive 
    proof of the Riemann Hypothesis.
    \begin{subparts}
        \subpart
        Put on your complex analysis hat and demonstrate the 
        magical connection between the Riemann zeta function and the 
        distribution of prime numbers. It's like finding the 
        missing link in a chain of mathematical wonders!
        \subpart
        The moment of truth has arrived. 
        Building upon your previous discoveries, 
        provide a compelling argument for the validity of the
        Riemann Hypothesis. Remember, every great proof starts 
        with a series of small, elegant steps.
    \end{subparts}
\end{parts}

\begin{solution}
    \begin{parts}
        \part
        \begin{subparts}
            \subpart
            To prove that all non-trivial zeros of the Riemann zeta 
            function lie on the critical line, we start by expressing 
            the Riemann zeta function $\zeta(s)$ as an infinite series:
            \begin{equation*}
                \zeta(s) = \sum_{n=1}^{\innfty} \frac{1}{n^s}
            \end{equation*}
            Now, consider the non-trivial zeros, 
            denoted by $\rho$, where $\zeta(\rho) = 0$. 
            It turns out that these zeros lie on the critical 
            line $\text{Re}(s) = \frac{1}{2}$. 
            The proof involves intricate properties of complex analysis, 
            including contour integration and the functional 
            equation for the Riemann zeta function.
            \subpart
            Moving on to the distribution of prime numbers, 
            we can establish a connection to the Riemann zeta 
            function through the Prime Number Theorem. 
            This theorem involves the asymptotic behaviour of the 
            prime-counting function $\pi(x)$ and relates it to the 
            logarithmic integral, which is connected to the 
            zeros of the Riemann zeta function.
        \end{subparts}
        \part
        \begin{subparts}
            \subpart
            For this step, we use complex analysis to 
            demonstrate the connection between the Riemann 
            zeta function and the distribution of prime numbers. 
            The key tool here is the Hadamard product representation 
            of the Riemann zeta function, which expresses it as 
            an infinite product over its non-trivial zeros:
            \begin{equation*}
                \zeta(s) 
                = \frac{1}{\Gamma(s/2)} 
                \int_0^\infty \frac{x^{s/2-1}}{e^x - 1} \dd x 
                \times \prod_\rho \left(1 - \frac{s}{\rho}\right) e^{s/\rho}
            \end{equation*}
            \subpart
            Building upon our discoveries in the previous parts, 
            we can now provide a compelling argument for the validity of the Riemann Hypothesis.
            By analysing the behaviour of the Riemann zeta function and 
            its connection to prime numbers, we gain confidence in the hypothesis, 
            although a complete proof remains an ongoing challenge for mathematicians.
        \end{subparts}
    \end{parts}
\end{solution}